%
% summe: ∑
% Umlaute: äöü ÄÖÜ
% ss: ß
% euro: €
%
\documentclass{beamer}

%\usepackage{beamerthemeshadow}
%\usepackage{beamerthemesplit}
%\usepackage{utf8}
%\usepackage{times,framed,soul,}
\usetheme{Frankfurt}
\usecolortheme{rose}
\useoutertheme{infolines}
\setbeamercovered{transparent}
\usepackage{textcomp,ucs,graphicx}
\usepackage[utf8x]{inputenc}
\usepackage[ngerman,english]{babel}
%\usepackage[LGR,T1,LHE,TS1,OT1,OMS]{fontenc}
%\usepackage[LGR]{fontenc}
% fürs euro-zeichen:
% % fürs sourcecode:
% \usepackage{listings}
% \lstset{
% 	language=Perl,
% 	extendedchars=false,
% 	inputencoding=utf8
% }

\title{Perl 6 - eine Einführung}
\author{Jens Rieks \texttt{parrot@jensbeimsurfen.de}}
\date{Austrian Perl Workshop 2005}

% Manual syntax highlighting
% taken from "Perl6_genau_jetzt.latex" by Ingo Blechschmidt <iblech@web.de>
\newcommand{\synfunc}   [1]{\color{blue!50!black}#1\color{black}}
\newcommand{\synstr}    [1]{\color{red!50!black}#1\color{black}}
\newcommand{\synvar}    [1]{\color{purple!50!black}#1\color{black}}
\newcommand{\synclass}  [1]{\color{green!50!black}#1\color{black}}
\newcommand{\syncomment}[1]{\color{blue!20!black}#1\color{black}}
\newcommand{\syncool}   [1]{\color{beamer@blendedblue}#1\color{black}}
\newcommand{\synoder}      {\ \ \color{black}$\vee$\ \ }
\newcommand{\hr}        {\rule[4pt]{\textwidth}{0.1pt}\\}

% some greek symbols
\newcommand{\textepsilon} {$\varepsilon$}
\newcommand{\textSigma} {$\varSigma$}

%\listfiles

\begin{document}

\section{Einführung}

\subsection{Willkommen!}
%%%%%%%%%%%%%%%%%%%%%%%%%%%%%%%%%%%%%%%%%%%%%%%%%%%%%%%%
\begin{frame}
	\frametitle{Willkommen!}
	\titlepage
\end{frame}


%%%%%%%%%%%%%%%%%%%%%%%%%%%%%%%%%%%%%%%%%%%%%%%%%%%%%%%%
% \subsection{Inhalt}
% \begin{frame}
% 	\frametitle{Einführung}
% 	\tableofcontents
% \end{frame}


%%%%%%%%%%%%%%%%%%%%%%%%%%%%%%%%%%%%%%%%%%%%%%%%%%%%%%%%
\subsection{Inhalt}
\begin{frame}
	\frametitle{Eine deutschsprachige Einführung in Perl 6}

	\begin{block}{Quellen}
	Die Präsentation ist eine Zusammenfassung der Perl 6 Synopsis-Dokumente.
	(weitere am Ende)
	\end{block}

	\pause
	
	\begin{block}{Ziel: Übersicht geben}
	über Dinge, die sich im Vergleich zu Perl 5 geändert haben
	\end{block}

	\pause
	
	\begin{block}{Beispiele}
	für neue Sprachkonstrukte präsentieren
	\end{block}

\end{frame}


% %%%%%%%%%%%%%%%%%%%%%%%%%%%%%%%%%%%%%%%%%%%%%%%%%%%%%%%%
% \subsection{Hintergrund}
% \begin{frame}
% 	\frametitle{Hintergrund}
% 
% 	Entstehungprozess, Entwicklung
% 	
% 	Anfang
% 	RFCs, Apocalypse, Exegeses
% 	Reihenfolge basiert auf Larry's Buch xxx
% \end{frame}


%%%%%%%%%%%%%%%%%%%%%%%%%%%%%%%%%%%%%%%%%%%%%%%%%%%%%%%%
%\part[Syntax]{Die Syntax von Perl 6}
\section{Syntax}

\begin{frame}
	\frametitle{Perl 6's Syntax}
	Teil 1 - Die Syntax von Perl 6
\end{frame}

\subsection{Grundlagen}
%%%%%%%%%%%%%%%%%%%%%%%%%%%%%%%%%%%%%%%%%%%%%%%%%%%%%%%%
\begin{frame}
	\frametitle{Hallo, Perl!}

	\begin{block}{Perl soll Perl bleiben}
	Daher: Die grundsätzliche Syntax bleibt erhalten\\
	\end{block}
	\ \\
	\uncover<2->{
		\input{code/syn/hallo1.pl.tex}
		\ \\
		\input{code/syn/hallo2.pl.tex}
	}
\end{frame}
%%%%%%%%%%%%%%%%%%%%%%%%%%%%%%%%%%%%%%%%%%%%%%%%%%%%%%%%
\begin{frame}
	\frametitle{Unterscheidung von Perl 5 und 6}

	\begin{block}{Abwärtskompatibiliät}
	Solange perl nicht mit Sicherheit sagen kann, ob
	Perl 5 oder Perl 6 Code vorliegt, geht der Parser
	davon aus, das es sich um Perl 5 handelt.
	\end{block}
	
	\begin{block}{Perl 6}
	"\texttt{use v6;}" ist der offizielle Weg, Perl 6 Code zu kennzeichnen
	\end{block}
	
	\begin{block}{Beispiele in dieser Präsentation}
	für Perl 5 Code benutze ich hauptsächlich "\texttt{print}", für Perl 6 "\texttt{say}"
	\end{block}
\end{frame}
%%%%%%%%%%%%%%%%%%%%%%%%%%%%%%%%%%%%%%%%%%%%%%%%%%%%%%%%
\begin{frame}
	\frametitle{Anweisungen}
	
	\begin{block}{Anweisungen}
	... enden weiterhin mit einem Semikolon \\
	\uncover<2->{... bei der letzten Anweisung eines Blockes ist es jedoch weiterhin optional}
	\ \\
	\ \\
	\uncover<3->{
	$\Rightarrow$ Das Semikolon ist ein Anweisungsseperator,\\
	kein Anweisungsterminator wie z.B. in C/C++ und Java.}
	\end{block}
\end{frame}

\subsection{Kommentare}
%%%%%%%%%%%%%%%%%%%%%%%%%%%%%%%%%%%%%%%%%%%%%%%%%%%%%%%%
\begin{frame}
	\frametitle{Kommentare}
	\input{code/syn/kommentare.pl.tex}
\end{frame}


\subsection{Variablen}
%%%%%%%%%%%%%%%%%%%%%%%%%%%%%%%%%%%%%%%%%%%%%%%%%%%%%%%%
\begin{frame}
	\frametitle{Variablen}
	\input{code/syn/vars1.pl.tex}
\end{frame}
\begin{frame}
	\frametitle{Variablen}
	\input{code/syn/vars2.pl.tex}
\end{frame}
\begin{frame}
	\frametitle{Variablen}
	\input{code/syn/vars3.pl.tex}
\end{frame}


\subsection{Otto und sein Hund}
%%%%%%%%%%%%%%%%%%%%%%%%%%%%%%%%%%%%%%%%%%%%%%%%%%%%%%%%
\begin{frame}
	\frametitle{Perl 5 Beispiel}
	\input{code/syn/otto1.pl.tex}
\end{frame}
\begin{frame}
	\frametitle{Perl 6 Beispiel (1)}
	\input{code/syn/otto2.pl.tex}
\end{frame}
\begin{frame}
	\frametitle{Perl 6 Beispiel (2)}
	\input{code/syn/otto3.pl.tex}
\end{frame}



\subsection{Schlüsselwörter}
%%%%%%%%%%%%%%%%%%%%%%%%%%%%%%%%%%%%%%%%%%%%%%%%%%%%%%%%
\begin{frame}
	\frametitle{Schlüsselworter}
	
	\begin{block}{Perl 5 und 6}
	my, our, sub
	\end{block}
	\begin{block}{neu in Perl 6}
	temp, let, state, class, method, \textellipsis
	\end{block}
\end{frame}

%%%%%%%%%%%%%%%%%%%%%%%%%%%%%%%%%%%%%%%%%%%%%%%%%%%%%%%%
\begin{frame}
	\frametitle{temp}
	\input{code/syn/temp1.pl.tex}
\end{frame}
% %%%%%%%%%%%%%%%%%%%%%%%%%%%%%%%%%%%%%%%%%%%%%%%%%%%%%%%%
% \begin{frame}
% 	\frametitle{temp}
% 	\input{code/syn/temp2.pl.tex}
% \end{frame}

%%%%%%%%%%%%%%%%%%%%%%%%%%%%%%%%%%%%%%%%%%%%%%%%%%%%%%%%
\begin{frame}
	\frametitle{Schlüsselworter}
	\input{code/syn/let1.pl.tex}
\end{frame}

%%%%%%%%%%%%%%%%%%%%%%%%%%%%%%%%%%%%%%%%%%%%%%%%%%%%%%%%
\begin{frame}
	\frametitle{state}
	\input{code/syn/state1.pl.tex}
\end{frame}
%%%%%%%%%%%%%%%%%%%%%%%%%%%%%%%%%%%%%%%%%%%%%%%%%%%%%%%%
\begin{frame}
	\frametitle{state}
	\input{code/syn/state2.pl.tex}
\end{frame}
\begin{frame}
	\frametitle{class und method}
	Wird ausführlicher im 3. Teil behandelt!
	\ \\
	\ \\
	\input{code/syn/class.pl.tex}
\end{frame}




\subsection{Programmfluß-Kontrolle}
%%%%%%%%%%%%%%%%%%%%%%%%%%%%%%%%%%%%%%%%%%%%%%%%%%%%%%%%
\begin{frame}
	\frametitle{Kontrollstrukturen}
	
	\begin{block}{Perl 5}
	if, for/foreach, while/until, \textellipsis
	\end{block}
	\begin{block}{Perl 6}
	if, loop/for, while/until, \textellipsis
	\end{block}
\end{frame}

% %%%%%%%%%%%%%%%%%%%%%%%%%%%%%%%%%%%%%%%%%%%%%%%%%%%%%%%%
% \begin{frame}
% 	\frametitle{if}
%  	\input{code/syn/if1.pl.tex}
% \end{frame}

%%%%%%%%%%%%%%%%%%%%%%%%%%%%%%%%%%%%%%%%%%%%%%%%%%%%%%%%
\begin{frame}
	\frametitle{for}
	\input{code/syn/for1.pl.tex}
	\input{code/syn/for2.pl.tex}
\end{frame}
%%%%%%%%%%%%%%%%%%%%%%%%%%%%%%%%%%%%%%%%%%%%%%%%%%%%%%%%
\begin{frame}
	\frametitle{for}
	\input{code/syn/for2.pl.tex}
	\input{code/syn/for3.pl.tex}
\end{frame}

%%%%%%%%%%%%%%%%%%%%%%%%%%%%%%%%%%%%%%%%%%%%%%%%%%%%%%%%
\begin{frame}
	\frametitle{C-ähnliche Schleifen}
	\input{code/syn/loop1.pl.tex}
	\input{code/syn/loop2.pl.tex}
\end{frame}
\begin{frame}
	\frametitle{"endlos"-Schleifen}
	\input{code/syn/loop3.pl.tex}
	\input{code/syn/loop4.pl.tex}
\end{frame}

%%%%%%%%%%%%%%%%%%%%%%%%%%%%%%%%%%%%%%%%%%%%%%%%%%%%%%%%
\begin{frame}
	\frametitle{while}
	\input{code/syn/while1.pl.tex}
	\input{code/syn/while2.pl.tex}
\end{frame}
%%%%%%%%%%%%%%%%%%%%%%%%%%%%%%%%%%%%%%%%%%%%%%%%%%%%%%%%
\begin{frame}
	\frametitle{do...while}
	\input{code/syn/while3.pl.tex}
	\input{code/syn/while4.pl.tex}
\end{frame}
%%%%%%%%%%%%%%%%%%%%%%%%%%%%%%%%%%%%%%%%%%%%%%%%%%%%%%%%
\begin{frame}
	\frametitle{given...when}
	\input{code/syn/given1.pl.tex}
\end{frame}




\subsection{Interpolation}
%%%%%%%%%%%%%%%%%%%%%%%%%%%%%%%%%%%%%%%%%%%%%%%%%%%%%%%%
\begin{frame}
	\frametitle{interpolierte Funktions-Aufrufe}
	\input{code/syn/interp1.pl.tex}
	\input{code/syn/interp2.pl.tex}
\end{frame}
\begin{frame}
	\frametitle{interpolierte Funktions-Aufrufe}
	\input{code/syn/interp3.pl.tex}
	\input{code/syn/interp4.pl.tex}
\end{frame}



% \subsection{Exceptions}
% %%%%%%%%%%%%%%%%%%%%%%%%%%%%%%%%%%%%%%%%%%%%%%%%%%%%%%%%
% \begin{frame}
% %  CATCH { 
% %       when Err::Danger { warn "fly away home"; }
% %   }
% % 
% % The $! object will also stringify to its text message if you match it against a pattern.
% %
% %  try {
% %      may_throw_exception();
% %   CATCH {
% %       when /:w I'm sorry Dave/ { warn "HAL is in the house."; }
% %   }
% % }
% \end{frame}


\subsection{Traits}
%%%%%%%%%%%%%%%%%%%%%%%%%%%%%%%%%%%%%%%%%%%%%%%%%%%%%%%%
\begin{frame}
	\frametitle{Unicode}
	\input{code/syn/traits1.pl.tex}
\end{frame}


\subsection{Unicode}
%%%%%%%%%%%%%%%%%%%%%%%%%%%%%%%%%%%%%%%%%%%%%%%%%%%%%%%%
\begin{frame}
	\frametitle{Unicode}
	
	Unicode
	\begin{itemize}
	\item wird bereits in Perl 5 teilweise unterstützt (use utf8;)
	\uncover<2->{
	\item \textellipsis aber nur sehr eingeschränkt, z.B. nicht in Sub-Namen
	}
	\uncover<3->{
	\item Perl 6 wird Unicode uneingeschränkt unterstützen
	}
	\uncover<4->{
	\item \textellipsis das Problem, dass Unicode-Zeichen in Dateinamen nicht\\
	portabel sind ist bislang jedoch noch ungelößt.
	}
	\end{itemize}
\end{frame}
%%%%%%%%%%%%%%%%%%%%%%%%%%%%%%%%%%%%%%%%%%%%%%%%%%%%%%%%
\begin{frame}
	\frametitle{Programme mit Unicode-Zeichen}
	
	Perl 5 und Unicode-Zeichen...
	
	\input{code/syn/waehrung1.pl.tex}
\end{frame}
%%%%%%%%%%%%%%%%%%%%%%%%%%%%%%%%%%%%%%%%%%%%%%%%%%%%%%%%
\begin{frame}
	\frametitle{Funktionen mit Unicode-Zeichen}
	
	Perl 6: Unicode-Zeichen sogar in Bezeichnernamen:
	
	\input{code/syn/waehrung2.pl.tex}
\end{frame}
%%%%%%%%%%%%%%%%%%%%%%%%%%%%%%%%%%%%%%%%%%%%%%%%%%%%%%%%
\begin{frame}
	\frametitle{Funktionen mit Unicode-Zeichen}
	
	In Perl 5 werden zwar auch einige Unicode-Zeichen in Sub-Namen akzeptiert,
	aber es sind recht wenige und es wird offiziell nicht unterstützt.
	
	\input{code/syn/summe1.pl.tex}
\end{frame}
\begin{frame}
	\frametitle{Funktionen mit Unicode-Zeichen}
	
	In Perl 6 wird Unicode standardmäßig unterstützt:
	
	\input{code/syn/summe2.pl.tex}
\end{frame}
\begin{frame}
	\frametitle{Funktionen mit Unicode-Zeichen}
	
	Perl 6 hat jede Menge coole Operatoren:
	
	\input{code/syn/summe3.pl.tex}
	\ \\
	Sie werden nun im nächsten Abschnitt behandelt.
\end{frame}


\section{Operatoren}

%%%%%%%%%%%%%%%%%%%%%%%%%%%%%%%%%%%%%%%%%%%%%%%%%%%%%%%%
\begin{frame}
	\frametitle{Perl 6's Operatoren}
	Teil 2 - Die Operatoren von Perl 6
\end{frame}

\subsection{Arten}
%%%%%%%%%%%%%%%%%%%%%%%%%%%%%%%%%%%%%%%%%%%%%%%%%%%%%%%
\begin{frame}
	\frametitle{Operator Arten}
	
	("Periodic Table of Operators")\\
	\ \\
	\uncover<2>{
	übersichtlicher ist es, wenn man nach mathematischen
	Aspekten kategorisiert:
	\begin{itemize}
	\item unäre Operatoren (prefix, postfix)
	\item binäre Operatoren (infix, circumfix)
	\item nulläre Operatoren
	\item tertiäre Operatoren
	\end{itemize}
	In dieser Reihenfolge werde ich die Operatoren nun vorstellen.
	}
\end{frame}

\subsection{unäre}
%%%%%%%%%%%%%%%%%%%%%%%%%%%%%%%%%%%%%%%%%%%%%%%%%%%%%%%%
\begin{frame}
	\frametitle{unäre Operatoren}
	\input{code/op/op-arten.pl.tex}
\end{frame}

\subsection{binäre}
%%%%%%%%%%%%%%%%%%%%%%%%%%%%%%%%%%%%%%%%%%%%%%%%%%%%%%%%
\begin{frame}
	\frametitle{binäre Operatoren}
	\input{code/op/op-arten2.pl.tex}
\end{frame}

\subsection{andere}
%%%%%%%%%%%%%%%%%%%%%%%%%%%%%%%%%%%%%%%%%%%%%%%%%%%%%%%%
\begin{frame}
	\frametitle{andere Operatoren}
	\input{code/op/op-arten3.pl.tex}
\end{frame}

\subsection{Syntax}
%%%%%%%%%%%%%%%%%%%%%%%%%%%%%%%%%%%%%%%%%%%%%%%%%%%%%%%%
\begin{frame}
	\frametitle{weitere Aufrufsmöglichkeiten}
	\input{code/op/aufruf1.pl.tex}
\end{frame}

\subsection{Beispiel}
%%%%%%%%%%%%%%%%%%%%%%%%%%%%%%%%%%%%%%%%%%%%%%%%%%%%%%%%
\begin{frame}
	\frametitle{Wolf im Schafspelz}
	\input{code/op/typen-konvertierung.pl.tex}
\end{frame}




\subsection{Zuweisung}
%%%%%%%%%%%%%%%%%%%%%%%%%%%%%%%%%%%%%%%%%%%%%%%%%%%%%%%%
\begin{frame}
	\frametitle{Zuweisungs-Operatoren}
	\input{code/op/zuweisung.pl.tex}
\end{frame}




\subsection{Bindungstest}
%%%%%%%%%%%%%%%%%%%%%%%%%%%%%%%%%%%%%%%%%%%%%%%%%%%%%%%%
\begin{frame}
	\frametitle{Bindungstest}
	\input{code/op/bind1.pl.tex}
\end{frame}




\subsection{Vergleiche}
%%%%%%%%%%%%%%%%%%%%%%%%%%%%%%%%%%%%%%%%%%%%%%%%%%%%%%%%
\begin{frame}
	\frametitle{Vergleichs-Operatoren}
	Vergleichs-Operatoren liefern  'true' oder 'false':
	\input{code/op/vergl1.pl.tex}
\end{frame}




\subsection{Wiederholungen}
%%%%%%%%%%%%%%%%%%%%%%%%%%%%%%%%%%%%%%%%%%%%%%%%%%%%%%%%
\begin{frame}
	\frametitle{Wiederholungs-Operatoren}
	\input{code/op/wiederholungs-ops.pl.tex}
\end{frame}
%%%%%%%%%%%%%%%%%%%%%%%%%%%%%%%%%%%%%%%%%%%%%%%%%%%%%%%%
\begin{frame}
	\frametitle{Wiederholungs-Operatoren}
	\input{code/op/wurf1.pl.tex}
\end{frame}


% \subsection{Zip}
% %%%%%%%%%%%%%%%%%%%%%%%%%%%%%%%%%%%%%%%%%%%%%%%%%%%%%%%%
% \begin{frame}
% %   for %ages.kv -> $name, $age {
% %       print $name, " is now ", $age;
% %   }
% \end{frame}



\subsection{Bereiche}
%%%%%%%%%%%%%%%%%%%%%%%%%%%%%%%%%%%%%%%%%%%%%%%%%%%%%%%%
\begin{frame}
	\frametitle{Bereichs-Operator}
	\input{code/op/bereichs-ops.pl.tex}
\end{frame}



\subsection{Kontext}
%%%%%%%%%%%%%%%%%%%%%%%%%%%%%%%%%%%%%%%%%%%%%%%%%%%%%%%%
\begin{frame}
	\frametitle{Kontext-Operatoren}
	\input{code/op/kontext-ops.pl.tex}
\end{frame}

% * Void Kontext
% Expects no value.

% * Item Kontext
% Expects a single value. A composite value returns a reference to itself in item context.

% * Boolean Kontext
% Expects a true or false value. This includes the traditional definitions of truth--where 0, undef, and the empty string are false and all other values are true--and values flagged with the properties true or false.

% * Integer Kontext
% Expects an integer value. Strings are treated as numeric and floating-point numbers are truncated.

% * Numeric Kontext
% Expects a number, whether it's an integer or floating-point, and whether it's decimal, binary, octal, hex, or some other base.

% * String Kontext
% Expects a string value. It interprets any information passed to it as a string of characters.

% * Objekt Kontext
% Expects an object, or more specifically, a reference to an object.

% * List context
% Expects a collection of values. Any single value in list context is treated as a one-element list.

% * Flattening list context
% Expects a list. Flattens out arrays and hashes into their component parts.

% * Non-flattening list context
% Expects a list. Treats arrays, hashes, and other composite values as discrete entities.

% * Lazy list context
% Expects a list, just like non-flattening list context, but doesn't require all the elements at once.

% * Hashlist context
% Expects a list of pairs. A simple list in hashlist context pairs up alternating elements.


\subsection{logische Operatoren}
%%%%%%%%%%%%%%%%%%%%%%%%%%%%%%%%%%%%%%%%%%%%%%%%%%%%%%%%
\begin{frame}
	\frametitle{Kontext-Operatoren}
	\input{code/op/logik1.pl.tex}
\end{frame}



\subsection{Smart-Match}
%%%%%%%%%%%%%%%%%%%%%%%%%%%%%%%%%%%%%%%%%%%%%%%%%%%%%%%%
\begin{frame}
	\frametitle{Smart-Match-Operatoren}
	\input{code/op/smart1.pl.tex}
\end{frame}
%%%%%%%%%%%%%%%%%%%%%%%%%%%%%%%%%%%%%%%%%%%%%%%%%%%%%%%%
\begin{frame}
	\frametitle{Smart-Match-Operatoren}
	\input{code/op/smart2.pl.tex}
\end{frame}



 
\subsection{Hyper}
%%%%%%%%%%%%%%%%%%%%%%%%%%%%%%%%%%%%%%%%%%%%%%%%%%%%%%%%
\begin{frame}
	\frametitle{Hyper-Operatoren}
	\input{code/op/hyper1.pl.tex}
\end{frame}


\subsection{Reduktion}
%%%%%%%%%%%%%%%%%%%%%%%%%%%%%%%%%%%%%%%%%%%%%%%%%%%%%%%%
\begin{frame}
	\frametitle{logische Reduktions-Metaoperatoren}
	\textellipsis wenden den in den Klammern stehenden \\
	Operator auf alle Elmente einer Liste an:
	\input{code/op/reduce1.pl.tex}
\end{frame}



\subsection{Junktionen}
%%%%%%%%%%%%%%%%%%%%%%%%%%%%%%%%%%%%%%%%%%%%%%%%%%%%%%%%
\begin{frame}
	\frametitle{Glückspiele...}
	Angenommen, man darf drei mal Würfel, bis man eine 6 hat.
	\input{code/op/wurf2.pl.tex}
\end{frame}
%%%%%%%%%%%%%%%%%%%%%%%%%%%%%%%%%%%%%%%%%%%%%%%%%%%%%%%%
\begin{frame}
	\frametitle{Glückspiele...}
	Angenommen, bei drei Würfen muss genau eine 6 bei sein:
	\input{code/op/wurf3.pl.tex}
\end{frame}
%%%%%%%%%%%%%%%%%%%%%%%%%%%%%%%%%%%%%%%%%%%%%%%%%%%%%%%%
\begin{frame}
	\frametitle{Alterskontrolle}
	\input{code/op/junkt1.pl.tex}
\end{frame}
% Table: Junctions
% 
% Function 	Operator 	Relation 	Meaning
% all 	& 	AND 	Operation must be true for all values.
% any 	| 	OR 	Operation must be true for at least one value.
% one 	^ 	XOR 	Operation must be true for exactly one value.
% none 		NOT 	Operation must be false for all values.
%%%%%%%%%%%%%%%%%%%%%%%%%%%%%%%%%%%%%%%%%%%%%%%%%%%%%%%%
%	$a = +@array # Anzahl der Elemente
%	
% 	eigenschaften von variablen: traits und properties
% 			
% 	Junktionen
% 	all($a,$b)	any($a,$b)	one($a,$b)	not($a,$b)
% 	
% 	Junktionen sind Objekte.
% 	sie können verglichen werden:
% 	if all($a,$b)==none($c,$d) {...}
% 	Methoden sind:
% 	.values()
% 	.dump()
% 	.pick()
% \end{frame}

 
%%%%%%%%%%%%%%%%%%%%%%%%%%%%%%%%%%%%%%%%%%%%%%%%%%%%%%%%
\section{Objekte}

%%%%%%%%%%%%%%%%%%%%%%%%%%%%%%%%%%%%%%%%%%%%%%%%%%%%%%%%
% \subsection{Datenstrukturen}
%%%%%%%%%%%%%%%%%%%%%%%%%%%%%%%%%%%%%%%%%%%%%%%%%%%%%%%%
% \begin{frame}
% DATENSTRUKTUREN
%   $arrayref = [ "a", "b", "c" ];
%   $hashref = { "a" => 123, "b" => 1234, "c" => 12345 };
% \end{frame}

%%%%%%%%%%%%%%%%%%%%%%%%%%%%%%%%%%%%%%%%%%%%%%%%%%%%%%%%
% \subsection{Pairs}
%%%%%%%%%%%%%%%%%%%%%%%%%%%%%%%%%%%%%%%%%%%%%%%%%%%%%%%%
% \begin{frame}
%   $pair = 'key' => 'value';
%   $pair = :key('value');
%   @pairs = %hash;
% \end{frame}

%%%%%%%%%%%%%%%%%%%%%%%%%%%%%%%%%%%%%%%%%%%%%%%%%%%%%%%
\subsection{Code-Objekte}
%%%%%%%%%%%%%%%%%%%%%%%%%%%%%%%%%%%%%%%%%%%%%%%%%%%%%%%
\begin{frame}
	\frametitle{Code-Blöcke}
	\input{code/class/code1.pl.tex}
\end{frame}
%%%%%%%%%%%%%%%%%%%%%%%%%%%%%%%%%%%%%%%%%%%%%%%%%%%%%%%
\begin{frame}
	\frametitle{Packen}
	\input{code/class/code2.pl.tex}
\end{frame}
%%%%%%%%%%%%%%%%%%%%%%%%%%%%%%%%%%%%%%%%%%%%%%%%%%%%%%%
\begin{frame}
	\frametitle{Currying}
	\input{code/class/code3.pl.tex}
\end{frame}
%%%%%%%%%%%%%%%%%%%%%%%%%%%%%%%%%%%%%%%%%%%%%%%%%%%%%%%
\begin{frame}
	\frametitle{NEXT/LAST Blöcke}
	\input{code/class/code4.pl.tex}
\end{frame}


%%%%%%%%%%%%%%%%%%%%%%%%%%%%%%%%%%%%%%%%%%%%%%%%%%%%%%%
\begin{frame}
	\frametitle{Coroutines}
	\input{code/class/coro1.pl.tex}
\end{frame}


% %%%%%%%%%%%%%%%%%%%%%%%%%%%%%%%%%%%%%%%%%%%%%%%%%%%%%%%
% \subsection{Filehandle-Objekte}
% %%%%%%%%%%%%%%%%%%%%%%%%%%%%%%%%%%%%%%%%%%%%%%%%%%%%%%%
% \begin{frame}
% 	\frametitle{Filehandle-Objekte}
% 
% 	Ein filehandle ist ein ganz normales Objekt.
% 
% 	\input{code/class/.pl.tex}
% \end{frame}


%%%%%%%%%%%%%%%%%%%%%%%%%%%%%%%%%%%%%%%%%%%%%%%%%%%%%%%%
\subsection{Klassen}
%%%%%%%%%%%%%%%%%%%%%%%%%%%%%%%%%%%%%%%%%%%%%%%%%%%%%%%%
\begin{frame}
	\frametitle{Klassen in Perl 6}
	\input{code/class/class1.pl.tex}
\end{frame}
\begin{frame}
	\frametitle{Methoden mit einem Parameter}
	\input{code/class/class2.pl.tex}
\end{frame}
\begin{frame}
	\frametitle{Methoden mit mehreren Parametern}
	\input{code/class/class3.pl.tex}
\end{frame}
\begin{frame}
	\frametitle{Polymorphismus}
	\input{code/class/class4.pl.tex}
\end{frame}
\begin{frame}
	\frametitle{Membervariablen}
	\input{code/class/class5.pl.tex}
\end{frame}
\begin{frame}
	\frametitle{Membervariablen}
	\input{code/class/class6.pl.tex}
\end{frame}

% delegation, multi-method dispatch


% % %%%%%%%%%%%%%%%%%%%%%%%%%%%%%%%%%%%%%%%%%%%%%%%%%%%%%%%%
% \section{Rules}
% 
% %%%%%%%%%%%%%%%%%%%%%%%%%%%%%%%%%%%%%%%%%%%%%%%%%%%%%%%%
% \subsection{Perl5 - Regular Expressions}
% %%%%%%%%%%%%%%%%%%%%%%%%%%%%%%%%%%%%%%%%%%%%%%%%%%%%%%%%
% \begin{frame}
% \end{frame}
% 
% %%%%%%%%%%%%%%%%%%%%%%%%%%%%%%%%%%%%%%%%%%%%%%%%%%%%%%%%
% \subsection{Perl6 - Rules}
% %%%%%%%%%%%%%%%%%%%%%%%%%%%%%%%%%%%%%%%%%%%%%%%%%%%%%%%%
% \begin{frame}
% \end{frame}


%%%%%%%%%%%%%%%%%%%%%%%%%%%%%%%%%%%%%%%%%%%%%%%%%%%%%%%
\section{Ende}
\subsection{weiterführende Literatur}
\begin{frame}
Literatur:
\begin{itemize}
\item die "Apocalypse"-Dokumente
\item die "Synopsis"-Dokumente
\item das Buch "Perl 6 \& Parrot Essentials"
\item ... etliche Internet-Quellen
\item kleinere Teile dieser Präsentation stammen von http://barthazi.hu/perl6ujdonsagok.sxi
\end{itemize}
\end{frame}
\subsection{Das wars!}
\begin{frame}
	\frametitle{Das wars!}
	Vielen Dank fürs zuhören!
	\ \\
	\ \\
	http://perl.jensbeimsurfen.de/
\end{frame}

\end{document}
