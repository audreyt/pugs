\section{Einführung}

\subsection{Willkommen!}
%%%%%%%%%%%%%%%%%%%%%%%%%%%%%%%%%%%%%%%%%%%%%%%%%%%%%%%%
\begin{frame}
	\frametitle{Willkommen!}
	\titlepage
\end{frame}


%%%%%%%%%%%%%%%%%%%%%%%%%%%%%%%%%%%%%%%%%%%%%%%%%%%%%%%%
% \subsection{Inhalt}
% \begin{frame}
% 	\frametitle{Einführung}
% 	\tableofcontents
% \end{frame}


%%%%%%%%%%%%%%%%%%%%%%%%%%%%%%%%%%%%%%%%%%%%%%%%%%%%%%%%
\subsection{Inhalt}
\begin{frame}
	\frametitle{Eine deutschsprachige Einführung in Perl 6}

	\begin{block}{Quellen}
	Die Präsentation ist eine Zusammenfassung der Perl 6 Synopsis-Dokumente.
	(weitere am Ende)
	\end{block}

	\pause
	
	\begin{block}{Ziel: Übersicht geben}
	über Dinge, die sich im Vergleich zu Perl 5 geändert haben
	\end{block}

	\pause
	
	\begin{block}{Beispiele}
	für neue Sprachkonstrukte präsentieren
	\end{block}

\end{frame}


% %%%%%%%%%%%%%%%%%%%%%%%%%%%%%%%%%%%%%%%%%%%%%%%%%%%%%%%%
% \subsection{Hintergrund}
% \begin{frame}
% 	\frametitle{Hintergrund}
% 
% 	Entstehungprozess, Entwicklung
% 	
% 	Anfang
% 	RFCs, Apocalypse, Exegeses
% 	Reihenfolge basiert auf Larry's Buch xxx
% \end{frame}
